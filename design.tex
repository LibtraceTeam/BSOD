
\section{Visual Design}
\label{visual}

\subsection{Layout}

Represents traffic only on one link so shows traffic traversing from one face to the opposite face of a cube (?)

Normal use seems to be with the two faces on opposite sides of the screen.  Little use for moving PoV.

\subsection{Visual Elements}

Packet colours, sizes, textures, speeds.

endpoints of connections.

\section{Software Design}
\label{software}

\subsection{Server}

Libtrace

Modules
	Colour modules
	Direction modules
	Layout modules
Modules API


\subsection{Client}

% Here's some example text by stj2

BuNG Engine

Various visualisation toolkits exist which aid in the development of 3d
visualisation software. Open source examples of such are Open Scene Graph
\cite{web:osg} and The Visualization Toolkit \cite{web:vtk}. The requirements
of the client were simple and did not require such large toolkits. The client
requires a framework to display dynamic data quickly and use a single TCP
connection with a custom protocol while remaining system independent. These
requirements were fulfilled by the Bung Engine \cite{web:bung}, a simple 3d
engine created initially by two undergraduate students. Using this engine
meant there was no time spent learning a new toolkit, allowing the initial
version of the client to be developed rapidly.

Use of GPU to improve performance

The Bung Engine can make use of the industry standard OpenGL 3d hardware
acceleration layer or Microsoft's Direct3D library. Both provide an interface
to the 3d acceleration card required to run BSOD at interactive framerates.
The client is designed to use the card's GPU to perform as much processing as
possible. This is limited somewhat by the rate at which data changes.
Traditionally geometry in a 3d visualisation can be compiled and sometimes
stored in the 3d acceleration card's memory to speed up rendering. This is not
possible with BSOD, because such methods require the data to be static, which
is not the case for data in BSOD.

Each flow in BSOD is batched together into one transformation and one drawing
call to improve performance, this means that the CPU does not need to
translate and rotate every packet in every flow and that the data is sent to
the 3d acceleration card in the most efficient manner.


Interaction

Client Server Comms


